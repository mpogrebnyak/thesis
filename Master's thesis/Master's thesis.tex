\documentclass[12pt, a4paper]{extarticle}
\usepackage{amsfonts}
\usepackage[T2A]{fontenc}
\usepackage[utf8]{inputenc}
%\usepackage{mathtext}  
\usepackage{amsmath, amsfonts, amssymb}
\usepackage[russian]{babel}
\usepackage[body={17.5cm, 23.5cm},left=3cm, top=2cm, right=2cm]{geometry}
\usepackage{graphicx}
\usepackage{blindtext}
\usepackage{fancyhdr}
\usepackage{graphicx}
\usepackage{ragged2e}
\usepackage{epigraph}
\usepackage{misccorr}  
\usepackage{indentfirst} 
\usepackage{amsmath}
\usepackage{tabularx} 

\usepackage{fancyhdr} 
\usepackage{color}

%\parindent{1.25cm} 
\graphicspath{images/}
\setcounter{tocdepth}{6}
\newcommand{\eps}{\varepsilon}
\newcommand{\re}{\operatorname{Re}}
\newcommand{\im}{\operatorname{Im}}
\DeclareMathOperator{\sgn}{sgn}
\renewcommand{\labelitemi}{$-$}
\renewenvironment{itemize}[1][{---\hfil}]{\begin{list}{#1}{\topsep=0pt\parsep=0pt plus 1pt\itemsep=\parsep\leftmargin=0pt \itemindent=\parindent}\addtolength{\itemindent}{\labelwidth}}{\end{list}}

\numberwithin{equation}{section} 

\newtheorem{attachment}{\hspace{12cm}  Приложение}
\renewcommand{\theattachment}{\Alph{attachment}}
%\renewcommand{\newtheorem}{\Alph{attachment}}
%\newtheorem{Conjecture}{Conjecture}[section]

\usepackage{tocloft}
\renewcommand{\cftsecleader}{\cftdotfill{\cftdotsep}}

\begin{document}
\thispagestyle{empty} 
\medskip 

\begin{center} 
	\textbf{МИНОБРНАУКИ РОССИИ\\ 
		\vspace{0.5cm} 
		Федеральное государственное бюджетное образовательное\\ 
		учреждение высшего образования\\ 
		«Ярославский государственный университет им. П.Г. Демидова»}\\ 
	\vspace{0.5cm} 
	{Кафедра математического моделирования}\\ 
	\vspace{1.5cm} 
	
\end{center}
\begin{flushright} 
	Сдано на кафедру\\
	« 
	\underline{\phantom{aaa}} 
	» 
	\underline{\phantom{aaaaaaaaaaaaa}} 2018 г.\\ 
	Заведующий кафедрой\\
	\underline{\phantom{aaa}д. ф.-м. н., профессор\phantom{aaa}}\\ 
	\vspace{0.1cm} 
	\underline{\phantom{aaaaaaaaaaaaa}} С.А. Кащенко
\end{flushright}
\vspace{3cm} 
\begin{center} 
	Выпускная квалификационная работа\\ 
	\vspace{0.5cm} 
	\textbf{Компьютерное моделирование движения физических объектов}\\ 
	\small{(Направление подготовки бакалавров 01.03.02 Прикладная математика и информатика)}
	\vspace{3cm} 
\end{center} 

\begin{flushright} 
	Научный руководитель\\ 
	\underline{\phantom{aaa}канд. ф-м. н., доцент\phantom{aaa}}\\ 
	\vspace{0.1cm} 
	\underline{\phantom{aaaaaaaaaaaaa}} И.С. Кащенко\\ 
	« 
	\underline{\phantom{aaa}} 
	» 
	\underline{\phantom{aaaaaaaaaaaaa}} 2018 г.\\ 
	\vspace{0.5cm} 
	Студент группы \underline{\phantom{a}ПМИ-42БО\phantom{a}}\\ 
	\vspace{0.1cm} 
	\underline{\phantom{aaaaaaaaaaaaa}} М.А. Погребняк\\ 
	« 
	\underline{\phantom{aaa}} 
	» 
	\underline{\phantom{aaaaaaaaaaaaaa}}2018 г.\\ 
	\vspace{1cm} 
\end{flushright} 
\begin{center} 
	Ярославль 2018 г.
	\vspace{-1cm}  
\end{center} 


\justify 
\setlength{\parindent}{1.25cm} 
\newpage 
\thispagestyle{empty} 
\setcounter{page}{2} 
\section*{Реферат}
\vspace{\baselineskip}	
Объем работы 21 страница, 4 главы, 19 рисунков, 9 источников, 1 приложение.

\textbf{Математическая модель, динамика движения, центр масс, система дифференциальных уравнений, метод Рунге-Кутты}

Объектами исследования являются физические объекты, а именно, многоугольник в плоскости и многогранник в пространстве.

Цель работы – построение математичкой модели движения физических объектов в плоскости и пространстве с дальнейшей её реализацией компьютерными технологиями.                                          

В результате работы были получены две модели, которые описывают динамику движения физических объектов на плоскости и в пространстве, а также была написана программа на языках программирования $C\#$ и $Python$, которая численно решает полученные модели и отображает полученную динамику.

\newpage

\setcounter{page}{2}

%\thispagestyle{empty} 
\tableofcontents 
\newpage 

\section*{Введение}
\addcontentsline{toc}{section}{Введение}
\epigraph{\textit{Так много в математике физики, как много в физике математики, и я уже перестаю находить разницу между этими науками}}
{-- Альберт Эйнштейн}

Физика за все время своего существования разгадала не мало тайн и загадок природы и люди научилась применять открытые и изученные законы с пользой для человека. Но математика как наука сформировалась первой, и по мере развития физических знаний математические методы находили всё большее применение в физических исследованиях. 

За время своего существования математика накопила богатейший инструмент для исследования окружающего мира. Математические формулы и теоремы вместе с их доказательством дают представление о множестве задач, которые можно с их помощью решить. Все отрасли современной науки тесно связаны между собой, поэтому математика и физика не исключение и их связь определяется прежде всего наличием общей предметной области, изучаемой ими, хотя и с различных точек зрения и выражается во взаимодействии их идей и методов. Физическая наука ставит задачи и создаёт необходимые для их решения математические идеи и методы, которые в дальнейшем служат базой для развития математической теории. А развитая математическая теория с её идеями и математическим аппаратом используется для анализа физических явлений, что часто приводит к новой физической теории, которая в свою очередь приводит к развитию физической картины мира и возникновению новых физических проблем. Само развитие физической теории опирается на имеющийся определённый математический аппарат, но последний совершенствуется и развивается по мере его использования в физике.

В настоящее время естественнонаучные изыскания в различных областях наук не возможно представить без широкого применения в них методов математического моделирования. Сущность методов математического моделирования состоит в замене исходного объекта его «подобием» – математической моделью и в дальнейшим исследованием полученной модели с помощью определённых алгоритмов. Метод математического моделирования сочетает в себе достоинства и теории, и эксперимента. Работа не с самим объектом (явлением, процессом), а с его моделью даёт возможность относительно быстро и без существенных затрат исследовать его свойства и поведение в любых ситуациях. Наиболее распространённый метод построения математического моделирования состоит в применении фундаментальных законов природы к конкретной ситуации. Эти законы общепризнанны, многократно подтверждены опытом и служат основой множества научно-технических достижений. Математическое моделирование участвует в любом расчёте физических процессов при этом составляется математическая модель - система уравнений, описывающая зависимости между физическими величинами при некоторых упрощающих допущениях.

Физические процессы описываются, как правило системой дифференциальных уравнений, для решения которой применяют различные численные методы (модели). Дифференциальные уравнения помогают решать различные задачи не только в математике, но и в физике, биологии, экономике и других науках, и сферах деятельности человека. Модель дифференциальных уравнений является примером математической модели, применяемой при решении задач.

Характеризуя физику с математикой как метод проникновения в тайны природы, можно сказать, что основным путём применения этого метода является формирование и изучение математических моделей реального мира. Здесь, может быть, уместно вспомнить слова А. Пуанкаре: "Математика - это искусство давать разным вещам одно наименование ". Эти слова являются выражением того, что математика изучает одним методом, с помощью математической модели, различные явления действительного мира. Изучая какие-либо физические явления, исследователь прежде всего создаёт его математическую идеализацию или, другими словами, математическую модель, то есть, пренебрегая второстепенными характеристиками явления, он записывает основные законы, управляющие этим явлением, в математической форме и очень часто эти законы можно выразить в виде дифференциальных уравнений. Для составления математической модели в виде дифференциальных уравнений нужно, как правило, знать только локальные связи и не нужна информация обо всем физическом явлении в целом.

Математическая модель даёт возможность изучать явление в целом, предсказать его развитие, делать количественные оценки изменений, происходящих в нем с течением времени. Теория дифференциальных уравнений в настоящее время представляет собой исключительно богатый содержанием, быстро развивающийся раздел математики, тесно связанный с другими областями математики с ее приложениями.
\newpage

\section{Постановка задачи} 

Используя законы механики, составить математическую модель движения физических объектов. В качестве физических объектов рассматриваются плоский многоугольник в плоскости и объёмный многогранник в пространстве. Данная модель должна представлять собой набор систем дифференциальных уравнений, которые описываю падение объекта на твёрдую поверхность. Полученную модель необходимо смоделировать, используя компьютерные технологии, то есть написать программу, которая решает полученные системы дифференциальных уравнений, используя численные методы, а также вычисляет любые другие необходимые параметры, которые требуются для полного решения поставленной задачи.  

\newpage

\section{Общая модель движения физических объектов}
 

\subsection{Движение многоугольника в плоскости}
      

\subsection{Движение многогранника в пространстве}


\section{Обратный переход от центра масс к угловым координатам}
 
 
\section{Реализация} 

  

\newpage
\section*{Заключение}
\addcontentsline{toc}{section}{Заключение}

Исходя из проведённых исследований, можно сделать вывод, что построение математической модели какого-либо физического процесса не тривиальная задача. Из множества физических сил и характеристик следует выбрать только тот минимум, который будет наиболее точно соответствовать и моделировать необходимый процесс. В ходе работы удалось получить две фундаментальные формулы, которые наиболее точно моделируют динамику движения объектов как в плоскости \eqref{main}, так и в пространстве \eqref{main_3d}. Так же на основе этих моделей была написана программа, которая численно решает эти модели и рисует полученную динамику.

Усложнение динамики не сильно усложнит ни системы \eqref{main}, \eqref{main_3d}, ни программу, поэтому можно считать их фундаментальными, и на их основе построить более сложную динамику движения физических объектов.

\newpage

\begin{thebibliography}{**}
	  
	\bibitem{Gravity}
	Цзю Х., Гоффман В. Гравитация и относительность. 1965.
	\bibitem{Polygon}
	Выгодский М.Я. Справочник по элементарной математике. 2001.
	\bibitem{Course}
	Погребняк М.А. Курсовая работа по теме "Компьютерное моделирование физических процессов". 2017.
	\bibitem{Center_of_mass}
	 Прядко Ю.Г., Караваев В.Г. Теоретическая механика. Геометрия масс. 2006.
	\bibitem{Moment_of_inertia}
	Фейнман Р., Лейтон Р., Сэндс М. Фейнмановские лекции по физике T. 2. 2016.
 	\bibitem{Runge_Kutta}
	 Бахвалов Н.С.  Численные методы. 1975. 
	 \bibitem{Geometry}
	 Яблокова С.И.   Лекции по курсу "Аналитическая геометрия". 2004. 
	\bibitem{Space_Geometry}
	Александров А. Д., Вернер А. Л., Рыжик В. И.  Стереометрия. Геометрия в пространстве. 1998. 
	\bibitem{Refactoring}
	Мартин Ф. Рефакторинг. Улучшение существующего кода. 2008.
\end{thebibliography}

\end{document}