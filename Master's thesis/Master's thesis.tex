\documentclass[12pt, a4paper]{extarticle}
\usepackage{amsfonts}
\usepackage[T2A]{fontenc}
\usepackage[utf8]{inputenc}
%\usepackage{mathtext}  
\usepackage{amsmath, amsfonts, amssymb}
\usepackage[russian]{babel}
\usepackage[body={17.5cm, 23.5cm},left=3cm, top=2cm, right=2cm]{geometry}
\usepackage{graphicx}
\usepackage{blindtext}
\usepackage{fancyhdr}
\usepackage{graphicx}
\usepackage{ragged2e}
\usepackage{epigraph}
\usepackage{misccorr}  
\usepackage{indentfirst} 
\usepackage{amsmath}
\usepackage{tabularx} 

\usepackage{fancyhdr} 
\usepackage{color}

%\parindent{1.25cm} 
\graphicspath{images/}
\setcounter{tocdepth}{6}
\newcommand{\eps}{\varepsilon}
\newcommand{\re}{\operatorname{Re}}
\newcommand{\im}{\operatorname{Im}}
\DeclareMathOperator{\sgn}{sgn}
\renewcommand{\labelitemi}{$-$}
\renewenvironment{itemize}[1][{---\hfil}]{\begin{list}{#1}{\topsep=0pt\parsep=0pt plus 1pt\itemsep=\parsep\leftmargin=0pt \itemindent=\parindent}\addtolength{\itemindent}{\labelwidth}}{\end{list}}

\numberwithin{equation}{section} 

\newtheorem{attachment}{\hspace{12cm}  Приложение}
\renewcommand{\theattachment}{\Alph{attachment}}
%\renewcommand{\newtheorem}{\Alph{attachment}}
%\newtheorem{Conjecture}{Conjecture}[section]

\usepackage{tocloft}
\renewcommand{\cftsecleader}{\cftdotfill{\cftdotsep}}

\begin{document}
\thispagestyle{empty} 
\medskip 

\begin{center} 
	\textbf{МИНОБРНАУКИ РОССИИ\\ 
		\vspace{0.5cm} 
		Федеральное государственное бюджетное образовательное\\ 
		учреждение высшего образования\\ 
		«Ярославский государственный университет им. П.Г. Демидова»}\\ 
	\vspace{0.5cm} 
	{Кафедра математического моделирования}\\ 
	\vspace{1.5cm} 
	
\end{center}
\begin{flushright} 
	Сдано на кафедру\\
	« 
	\underline{\phantom{aaa}} 
	» 
	\underline{\phantom{aaaaaaaaaaaaa}} 2018 г.\\ 
	Заведующий кафедрой\\
	\underline{\phantom{aaa}д. ф.-м. н., профессор\phantom{aaa}}\\ 
	\vspace{0.1cm} 
	\underline{\phantom{aaaaaaaaaaaaa}} С.А. Кащенко
\end{flushright}
\vspace{3cm} 
\begin{center} 
	Выпускная квалификационная работа\\ 
	\vspace{0.5cm} 
	\textbf{Компьютерное моделирование движения физических объектов}\\ 
	\small{(Направление подготовки бакалавров 01.03.02 Прикладная математика и информатика)}
	\vspace{3cm} 
\end{center} 

\begin{flushright} 
	Научный руководитель\\ 
	\underline{\phantom{aaa}канд. ф-м. н., доцент\phantom{aaa}}\\ 
	\vspace{0.1cm} 
	\underline{\phantom{aaaaaaaaaaaaa}} И.С. Кащенко\\ 
	« 
	\underline{\phantom{aaa}} 
	» 
	\underline{\phantom{aaaaaaaaaaaaa}} 2018 г.\\ 
	\vspace{0.5cm} 
	Студент группы \underline{\phantom{a}ПМИ-42БО\phantom{a}}\\ 
	\vspace{0.1cm} 
	\underline{\phantom{aaaaaaaaaaaaa}} М.А. Погребняк\\ 
	« 
	\underline{\phantom{aaa}} 
	» 
	\underline{\phantom{aaaaaaaaaaaaaa}}2018 г.\\ 
	\vspace{1cm} 
\end{flushright} 
\begin{center} 
	Ярославль 2018 г.
	\vspace{-1cm}  
\end{center} 


\justify 
\setlength{\parindent}{1.25cm} 
\newpage 
\thispagestyle{empty} 
\setcounter{page}{2} 
\section*{Реферат}
\vspace{\baselineskip}	

\newpage

\setcounter{page}{2}

%\thispagestyle{empty} 
\tableofcontents 
\newpage 

\section*{Введение}
\addcontentsline{toc}{section}{Введение}
\epigraph{\textit{Так много в математике физики, как много в физике математики, и я уже перестаю находить разницу между этими науками}}
{-- Альберт Эйнштейн}

\newpage

\section{Постановка задачи} 
  

\newpage

\section{Простая модель следования за лидером}

\begin{equation*}
\ddot{x}(t) = d (\dot{x}(t-\tau)-\dot{x}(t) - \lambda).
\end{equation*}

\begin{equation*}
\begin{split}
&\ddot{x}_1(t) = d (\dot{x}_0(t-\tau)-\dot{x}_1(t) - \lambda), \\ 
&\ddot{x}_2(t) = d (\dot{x}_1(t-\tau)-\dot{x}_2(t) - \lambda), \\
&\ldots \\
&\ddot{x}_N(t) = d (\dot{x}_{N-1}(t-\tau)-\dot{x}_N(t) - \lambda).
\end{split}
\end{equation*}

\begin{equation*}
\begin{split}
&\ddot{u}(t) = d (\dot{u}(t-\tau)-\dot{u}(t) - \lambda), \\ 
&\ddot{v}(t) = d (\dot{u}(t-\tau)-\dot{v}(t) - \lambda), \\
&\ddot{w}(t) = d (\dot{v}(t-\tau)-\dot{w}(t) - \lambda).
\end{split}
\end{equation*}

\begin{table}[h]
	\caption{Физическое значение параметров}
	\label{parameters}
	\begin{center}
		\begin{tabularx}{\textwidth}{p{0.15\linewidth}p{0.85\linewidth}}
			
			\hline
			\rule{0cm}{0,5cm}
			Параметр &  Физическое значение \\ 
			[3pt]\hline
			-- & -- \\
			\hline
		\end{tabularx}
	\end{center}
\end{table}

 
\section{Реализация} 
 

\newpage
\section*{Заключение}
\addcontentsline{toc}{section}{Заключение}

\newpage

\begin{thebibliography}{**}
	  
	\bibitem{Gravity}
	Цзю Х., Гоффман В. Гравитация и относительность. 1965.
	\bibitem{Polygon}
	Выгодский М.Я. Справочник по элементарной математике. 2001.
	\bibitem{Course}
	Погребняк М.А. Курсовая работа по теме "Компьютерное моделирование физических процессов". 2017.
	\bibitem{Center_of_mass}
	 Прядко Ю.Г., Караваев В.Г. Теоретическая механика. Геометрия масс. 2006.
	\bibitem{Moment_of_inertia}
	Фейнман Р., Лейтон Р., Сэндс М. Фейнмановские лекции по физике T. 2. 2016.
 	\bibitem{Runge_Kutta}
	 Бахвалов Н.С.  Численные методы. 1975. 
	 \bibitem{Geometry}
	 Яблокова С.И.   Лекции по курсу "Аналитическая геометрия". 2004. 
	\bibitem{Space_Geometry}
	Александров А. Д., Вернер А. Л., Рыжик В. И.  Стереометрия. Геометрия в пространстве. 1998. 
	\bibitem{Refactoring}
	Мартин Ф. Рефакторинг. Улучшение существующего кода. 2008.
\end{thebibliography}

\end{document}